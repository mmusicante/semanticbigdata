

Nowadays, large volumes of graph-structure data are widespread  in many different domains such as social network, scientific computing, telecoms network, linked open data, organization management, finance chains and so on (\cite{Poul13}).

As discussed in \cite{Poul13}, graph-structured data differs from ordinary big data in its deeper focus on relationships between  concepts or entities.
 Relationships are as important as entities, and we need not only to  model attributes or constraints on them but also to easy query and analysis over them.
 
 In this project we consider a data graph distributed in many different machines.
 One machine contains the specification of this graph (schema + constraints).
 
 Our goals are:
 
 \begin{enumerate}
 \item Choose and extend a schema language (possibly including some constraints)
 \item Use a resolver to infer new constraints; test consistency.
 \item Propose a user-friendly query language over the data graph. First step:  a datalog-like language. 
 \item Consider fragmentation/distribution according to usual queries.
 \item Propose  query execution plans for efficient query evaluation.
 \end{enumerate}
