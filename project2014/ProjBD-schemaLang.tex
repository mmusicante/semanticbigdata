
The graph schema is described by using a descriptive logic  (DL), to be translated into RDFS or OWL for implementation.
In our context, the schema is a Tbox (description of concepts, relationships and some kinds of constraints).
The Abox is the graph, basically described by concepts and relationships.
Tbox is not distributed while Abox is.

We start by considering the \dl\ fragment.


\subsection{\dl}

   The family of \dl\ has been designed for capturing the main modelling primitives of conceptual data model such as E-R or UML while remaining  tractable in the presence of
inclusion statements and a certain form of negation (\cite{AMRRS12}).

\begin{itemize}
\item The constructs allowed in \dl\ are unqualified existential restrictions on roles ($\exists R$)  and on inverse roles ($\exists R^{-}$) and the negation.
\item The axioms allowed in a Tbox of \dl\ are concept inclusion statements of the form $B \sqsubseteq C $  or  $B \sqsubseteq \neg C $ where $B$ and $C$ are atomic concepts
or existential  restriction ($\exists R$ or $\exists R^{-}$).
\item Negation is only allowed in the right-hand side of inclusions.
\item We have 2 types of inclusion:  negative inclusion (NI) and positive inclusion (PI).
\item A \dl\ schema may contain axioms corresponding to those proposed in RDFS. Besides it may contain other axioms of three kinds:
PI, NI and key constraints which are not expressible in RDFS (see OWL, so).


\item There are two dialects of \dl\, namely : 

\begin{itemize}
\item A \dlr: Tbox allows role inclusion statements $P \sqsubseteq Q $  or  $P \sqsubseteq \neg Q$ where $P$ and $Q$ are atomic  roles or inverse atomic roles. Thus it is obtained
by extending axioms of RDFS with PI and NI.
\item A \dlf:  Tbox allows functionality statements on roles of the form $funct P$ of $funct P^-$. To express keys.
Thus it is obtained by extending axioms of RDFS with key constraints, PI and NI, but \textit{excluding inclusion between properties}. Due to this exclusion, RDFS is not included in \dlf.
\end{itemize}

\item Remark: Why not simply extend RDFS with three kinds of axioms? Because functional constraints interact with inclusion constraints in intricate way.

\item Two fundamental differences between query answering in the context of RDFS and \dl\ knowledge bases:
\begin{itemize}
\item Inconsistency:  RDFS does not permit negation, but \dl\ allows some kind of negation - so, testing consistency is important.
\item Incompleteness: RDFS rules are safe (allowing the simple bottom-up algorithm to answer queries). However axioms in \dl\ may correspond to unsafe rules (cf. blanc nodes). We have to use top-down approach for evaluating queries !!!! 
\end{itemize}

\item Due to all these remarks... We start by using \dlr
\end{itemize}

\subsection{Schema}

The schema describes a graph. For defining a schema as a DL knowledge base we need:

\begin{itemize}
\item  A vocabulary composed of a set  \textbf{C} of atomic concepts $A, B, \dots$,
a set \textbf{R} of atomic roles $P, Q, R, \dots$ and a set of atomic individuals $a, b, c \dots$.


\item A set of constructs  used for building  tuples, complex concepts and roles  from atomic individuals, concepts and roles.
\item A language of axioms that can be stated for constraining  the vocabulary  in order to express domain constraints.

\item Atomic individuals are all names used to denote
singular entities (be they persons, objects or anything else) in our
domain of interest. 

\item If $a_1, \dots, a_n$ are atomic individuals, then $\tuple{a_1, \dots, a_n} $ is a n-ary individual tuple. In this document we only deal with binary tuples.

\item A language of axioms 
that can be stated for constraining the vocabulary in order to
express domain constraint.

\end{itemize}



\begin{definition}[\dlm\ Syntax] 
Let $A \in$   \textbf{C} be an atomic concept name, 
$P \in $ \textbf{C} be an atomic role and $P^-$ the inverse of  $P$.
In  the abstract syntax:

\begin{center}
\begin{tabular}{lcl}
$ B :: =  A \mid \exists R$ 	&\hspace{1cm}		&$ R ::= P \mid P^-$\\
$C ::= B \mid \neg B$ 		& 							&$ E ::= R \mid \neg R$ \\
\end{tabular}
\end{center}

\noindent
 $B$ denotes a \textit{basic concept}, \ie,  an atomic concept or a concept of 
 the form $\exists R$ and $R$ denotes a \textit{basic role}, \ie, a role that is either an 
 atomic role or the inverse of an atomic role. 
  Finally, $C$ denotes a \textit{}(general) concept, which can be a basic concept or its negation, 
  whereas $E$ denotes a \textit{(general) role}, which can be a basic role or its negation.

\end{definition}

A \dlm\ knowledge base ($\K = (\T,\A)$)  has two components: a TBox $\T$, used to represent intensional knowledge, and a ABox $\A$, used
to represent  extensional knowledge.
A \dlm\ TBox is formed by a finite set of inclusion assertions of the forms:

\centerline{$ B \sqsubseteq C$ or $ R \sqsubseteq E$.}
 
Thus, 
 we allow general concepts or roles to occur on the right-hand side of inclusion assertions, whereas only basic concepts or roles may occur on the left-hand side of inclusion assertions.
 
 The following table gives the semantics of \dlm\ in terms of first order logic (FOL) formulas.
 \dlm\  has an expression power equivalent to \textsc{DL-lite}$_{A}$ (described in \cite{CDLLR06c}) without value-domains and functionality assertion.
As \textsc{DL-lite}$_{A}$, \dlm\ allows the use of concept and role attributes, but uses a different syntax.
 In other words, we use \dlr\ to which we add concept and role attributes, expressed via special roles and binary tuples, to obtain \dlm.
 This allows for  LOGSPACE query answering.


\small
\begin{longtable}{| p{2cm} | p{8.2cm} | p{5.2cm} |}
%\begin{longtable}{| p{3.5cm} | p{9.2cm} |p{1cm}|}
	\hline
	\centering \textbf{\dlm} & \centering \textbf{FOL} & \centering \textbf{Datalog$_0^{\pm}$}
	\\
	\endfirsthead
	\hline	
	\centering \textbf{\dlm} & \centering \textbf{FOL} & %\centering \textbf{RDFS/OWL}
	\\
	\endhead
	
	\hline
		$i:A$ or $A(i)$
		&
		$A(i)$
		&
	%	$<$ \texttt{i rdf:type A} $>$
	$p_A(c_i)$
	\\
		
	\hline
		$i P j$ or $P(i,j)$
		&
		$P(i,j)$
		&
	%	$<$\texttt{ i P j }$>$
	$p(c_i, c_j)$
	\\
	
	\hline
		%\cellcolor[gray]{0.9} 
		$[i,j] P k$ or $P([i,j],k)$
		&
		%\cellcolor[gray]{0.9} 
		$P(\pair{i,j},k)$
		&
		%\cellcolor[gray]{0.9} 
%		$<$\texttt{ [i,j] P k}$>$
      $p(c_i, c_j, c_k)$
	\\
	
	\hline
		$A_1 \sqsubseteq A_2$	
		&
		$\forall X.(A_1(X) \Rightarrow A_2(X))$
		&
%%		$<$\texttt{ $A_1$ rdfs:subClassOf $A_2$ }$>$
$p_{A_1}(X) \rightarrow p_{A_2}(X)$
	\\

	\hline
		%\cellcolor[gray]{0.9} 
		$P \sqsubseteq P_2$	
		&
		%\cellcolor[gray]{0.9} 
		$\forall X Y. (P_1(X,Y) \Rightarrow P_2(X,Y))$ or \newline
		$\forall X Y Z. (P_1(\pair{X,Y},Z) \Rightarrow P_2(\pair{X,Y},Z))$
		&
		%\cellcolor[gray]{0.9} 
%%		$<$\texttt{ $P_1$ rdfs:subPropertyOf $P_2$}$>$
      $p_1(X,Y) \rightarrow p_2 (X,Y)$   or   $p_1(X,Y,Z) \rightarrow p_2 (X,Y, Z)$
	\\

	\hline
		$\exists P \sqsubseteq A$	
		&
		$\forall X Y. (P(X,Y) \Rightarrow A(X))$
		&
%%		$<$\texttt{ P rdfs:domain A }$>$
$p(X,Y) \rightarrow p_A (X)$  
	\\

	\hline
		%\cellcolor[gray]{0.9} 
		$\exists P^{-} \sqsubseteq A$	
		&
		%\cellcolor[gray]{0.9} 
		$\forall X Y. (P(X,Y) \Rightarrow A(Y))$ or \newline
		$\forall X Y Z. (P(\pair{X,Y},Z) \Rightarrow A(Z))$
		&
		%\cellcolor[gray]{0.9} 
%%		$<$\texttt{ P rdfs:range A }$>$
$p(X,Y) \rightarrow p_A (Y)$   or   $p(X,Y,Z) \rightarrow p_A (Z)$
	\\
	
	\hline
		%\cellcolor[gray]{0.7} 
		$\exists P_1 \sqsubseteq P_2$
		&
		%\cellcolor[gray]{0.7} 
		$\forall X Y Z. (P_1(\pair{X,Y},Z) \Rightarrow P_2(X,Y))$
		&
		%\cellcolor[gray]{0.7} 
	%%	$<$\texttt{ $P_1$ rdfs:domain $P_2$ }$>$*
	$p_1(X,Y, Z) \rightarrow p_2 (X,Y)$  
	\\

	\hline
		Positive inclusions & &
	\\
	
	\hline
		$A \sqsubseteq \exists P$		
		&
		$\forall X.(A(X) \Rightarrow \exists Y. P(X,Y)) $
		&
%%		if $<$\texttt{ i rdf:type A }$>$ \newline
%%		then $<$\texttt{ i P \_:blank }$>$
$p_A(X) \rightarrow \exists Y~ p (X,Y)$ 
	\\
	
	\hline	
		%\cellcolor[gray]{0.7} 
		$P_1 \sqsubseteq \exists P_2$		
		&
		%\cellcolor[gray]{0.7} 
		$\forall X Y .(P_1(X,Y) \Rightarrow \exists Z. P_2(\pair{X,Y},Z)) $
		&
		%\cellcolor[gray]{0.7} 
%%		if		$<$\texttt{ i $P_1$ j }$>$ \newline
%%		then $<$\texttt{ [i,j] $P_2$ \_:blank }$>$*
$(p_1(X,Y) \Rightarrow \exists Z~p_2(X,Y,Z)) $
	\\
	
	\hline
		$A \sqsubseteq \exists P^{-}$		
		&
		$\forall X .(A(X) \Rightarrow \exists Y. P(Y,X)) $
		&
%%		if $<$\texttt{ i rdf:type A}$>$ \newline
%%		then $<$\texttt{ \_:blank P i }$>$
$p_A(X) \Rightarrow \exists Y~p(Y,X)) $
	\\
	
	\hline
		%\cellcolor[gray]{0.9} 
		$\exists P_1 \sqsubseteq \exists P_2$		
		&
		%cellcolor[gray]{0.9} 
		$\forall X Y. (P_1(X,Y) \Rightarrow \exists Z .P_2(X,Z)) $ \newline
		$\forall X Y W.(P_1(\pair{X,Y},W) \Rightarrow \exists Z. P_2(\pair{X,Y}, Z)) $
		&
		%\cellcolor[gray]{0.9} 
%%		if $<$\texttt{ i $P_1$ j }$>$ \newline
%%		then $<$\texttt{ i $P_2$ \_:blank }$>$ or \newline
%%		
%%		%\cellcolor[gray]{0.9} 
%%		if $<$\texttt{ [i,j] $P_1$ k }$>$ \newline
%%		then $<$ \texttt{[i,j] $P_2$ \_:blank }$>$
$ (p_1(X,Y) \rightarrow \exists Z ~p_2(X,Z)) $ \newline
		$p_1(X,Y,W) \rightarrow \exists Z~p_2(X,Y, Z)) $
	\\
	
	\hline	
		%\cellcolor[gray]{0.9} 
		$\exists P_1\sqsubseteq \exists P^{-}$		
		&
		%\cellcolor[gray]{0.9} 
		$\forall X Y. (P_1(X,Y) \Rightarrow \exists Z. P_2(Z,X)) $ \newline
		$\forall X  Y. (P_1(X,Y) \Rightarrow \exists Z Z'. P_2(\pair{Z,Z'},X)) $
		&
		%\cellcolor[gray]{0.9} 
%%		if $<$\texttt{ i R j }$>$ \newline
%%		then $<$\texttt{ \_:blank P i }$>$ or \newline
%%		
%%		%\cellcolor[gray]{0.9} 
%%		if $<$\texttt{ [i,j] R k }$>$ \newline
%%		then $<$\texttt{ [\_:b1,\_:b2] P i }$>$
		$ (p_1(X,Y) \Rightarrow \exists Z~ p_2(Z,X)) $ % \newline
		$ (p_1(X,Y) \Rightarrow \exists Z Z'~p_2(Z,Z',X)) $

	\\
	
	\hline
		$P \sqsubseteq P_1^{-}$ or $P^{-} \sqsubseteq P_1$		
		&
		$\forall X Y. (P(X,Y) \Rightarrow P_1(Y,X)) $
		&
%%		$<$\texttt{P owl:inverseOf R}$>$
$p(X,Y) \Rightarrow p_1(Y,X) $
	\\
	
	\hline	
		Negative inclusions & &
	\\
	
	\hline
		$A_1 \sqsubseteq \neg A_2$	
		&
		$\forall X. (A_1(X) \Rightarrow \neg A_2(X)) $
		&
%%		$<$\texttt{C owl:disjoinWith D} $>$
$\ p_{A_1}(X) \Rightarrow \neg p_{A_2}(X)) $
	\\
	
	\hline
		%\cellcolor[gray]{0.9} 
		$P_1 \sqsubseteq \neg P_2$	
		&
		%\cellcolor[gray]{0.9} 
		$\forall X  Y. (P_1(X,Y) \Rightarrow \neg P_2(X,Y)) $ \newline
		%\cellcolor[gray]{0.9} 
		$\forall X  Y  Z.(P_1(\pair{X,Y},Z) \Rightarrow \neg P_2(\pair{X,Y},Z)) $
		&
		%\cellcolor[gray]{0.9} 
%%		$<$\texttt{P owl:disjoinWith R} $>$
$\ p_1(X,Y) \rightarrow \neg p_2(X,Y) $ \newline
		%\cellcolor[gray]{0.9} 
		$p_1(X,Y,Z) \rightarrow \neg p_2(X,Y,Z)) $
	\\
	
	\hline
		$P_1^{-} \sqsubseteq \neg P_2$	or $P_1 \sqsubseteq \neg P_2^{-}$
		&
		$\forall X Y .(P_1(X,Y) \Rightarrow \neg P_2(Y,X)) $
		&
		$p_1(X,Y) \Rightarrow \neg p_2(Y,X)$
	\\
	
	\hline
		$A \sqsubseteq \neg \exists P$
		&
		$\forall X. (A(X) \Rightarrow \neg \exists Y. P(X,Y)) $
		&
%%		if $<$\texttt{ i rdf:type C }$>$ \newline
%%		then not $<$\texttt{ i P \_:blank }$>$
$p_A(X) \Rightarrow \neg \exists Y~p(X,Y)) $
	\\
	
	\hline
		%\cellcolor[gray]{0.9} 
		$A \sqsubseteq \neg \exists P^{-}$	
		&
		%\cellcolor[gray]{0.9} 
		$\forall X. (A(X) \Rightarrow \neg \exists Y. P(Y,X)) $ \newline
		%\cellcolor[gray]{0.9} 
		$\forall X. (A(X) \Rightarrow \neg \exists Y Z. P(\pair{Y,Z},X)) $
		&
%%		if $<$\texttt{ i rdf:type C }$>$ \newline
%%		then not $<$\texttt{ \_:blank P i }$>$ or \newline
%%		
%%		if $<$\texttt{ i rdf:type C }$>$ \newline
%%		then not $<$\texttt{ [\_:b1,\_:b2] P i }$>$
$p_A(X) \rightarrow \neg \exists Y~p(Y,X)) $ \newline
		%\cellcolor[gray]{0.9} 
		$p_A(X) \rightarrow \neg \exists Y Z~ p(Y,Z,X)) $
	\\
	
	\hline
		%\cellcolor[gray]{0.7} 
		$P_1 \sqsubseteq \neg \exists P_2$	
		&
		%\cellcolor[gray]{0.7} 
		$\forall X Y. (P_1(X,Y) \Rightarrow \neg \exists Z. P_2(\pair{X,Y},Z))$
		&
		$p_1(X,Y) \Rightarrow \neg \exists Z. p_2(X,Y,Z))$
	\\
	
	\hline
		%\cellcolor[gray]{0.9} 
		$\exists P_1 \sqsubseteq \neg \exists P_2$	
		&
		%\cellcolor[gray]{0.9} 
		$\forall X Y. (P_1(X,Y) \Rightarrow \neg \exists Z. P_2(X,Z)) $ \newline
		%\cellcolor[gray]{0.9} 
		$\forall X  Y Z .(P_1(\pair{X,Y},Z) \Rightarrow \neg \exists Z' . P_2(\pair{X,Y},Z'))$
		&
		$p_1(X,Y) \rightarrow \neg \exists Z~P_2(X,Z) $ \newline
		%\cellcolor[gray]{0.9} 
		$p_1(X,Y,Z) \rightarrow \neg \exists Z' ~p_2(X,Y,Z'))$
	\\
	
	\hline
		%\cellcolor[gray]{0.9} 
		$\exists P_1 \sqsubseteq \neg \exists P_2^{-}$	
		&
		%\cellcolor[gray]{0.9} 
		$\forall X  Y. (P_1(X,Y) \Rightarrow \neg \exists Z. P_2(Z,X)) $ \newline
		%\cellcolor[gray]{0.9} 
		$\forall X  Y. (P_1(X,Y) \Rightarrow \neg \exists Z  Z' . P_2(\pair{Z,Z'},X)) $
		&
		$p_1(X,Y) \rightarrow \neg \exists Z~ p_2(Z,X) $ \newline
		%\cellcolor[gray]{0.9} 
		$p_1(X,Y) \rightarrow \neg \exists Z  Z' ~p_2(Z,Z',X)) $
	\\
	
	\hline
		$\exists P \sqsubseteq \neg A$
		&
		$\forall X Y. (P(X,Y) \Rightarrow \neg A(X)) $
		&
%%		$<$ \texttt{P rdfs:domain B }$>$ \newline
%%		$<$ \texttt{B owl:disjoinWith C }$>$
$p(X,Y) \rightarrow \neg p_A(X) $
	\\
	
	\hline
		$\exists P^{-} \sqsubseteq \neg A$
		&
		$\forall X  Y. (P(X,Y) \Rightarrow \neg A(Y)) $
		&
%%		$<$ \texttt{P rdfs:range B }$>$ \newline
%%		$<$ \texttt{B owl:disjoinWith D }$>$

	$p(X,Y) \Rightarrow \neg p_A(Y)) $
	\\
	\hline
	
\end{longtable}


\begin{example}
Suppose an ABox with the following information concerning concepts Professor  and Course and the role Teaches.
Its information is grafically shown in grapXXX

\begin{center}
Professor(Robert), Professor(Martin), Professor(Smith)\\
Course(Math), Course(CS),
Teaches(Robert, Math),  Teaches(Martin, Math), Teaches(Robert, CS),  Teaches(Smith, CS)
\end{center}

Courses have an attribute concerning the number of hours.
We use the role Course$_{Hours}$  to represent it.

\begin{center}
 Course$_{Hours}$ (Math, 30),  Course$_{Hours}$(CS, 60),
\end{center}

The role Teaches has also an attribute which indicates the course level.

\begin{center}
Teaches$_{Level}$ (\pair{Robert, Math}, graduated), \\
Teaches$_{Level}$(\pair{Martin, Math}, undergraduated), $\dots$
\end{center}
\end{example}
