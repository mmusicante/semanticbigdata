

\documentclass{beamer}

\newcommand \com[1]{}


\usecolortheme{beaver}
\usetheme{CambridgeUS}
%\usetheme{Boadilla}

\title[Big Data]{Big Data\\
Preparing an oveview \\
}
%\subtitle{subtitle}
\author[Mirian Halfeld Ferrari]{Mirian Halfeld Ferrari}
\institute[LIFO]{LIFO}
\date{Mars 2014}

\setbeamerfont{block title}{size={}}
\setbeamercolor{titlelike}{parent=structure,bg=white}

\setbeamerfont*{frametitle}{size=\normalsize,series=\bfseries}
%\setbeamertemplate{navigation symbols}{}
%\setbeamertemplate{blocks}[shadow=true]

% Standard packages
%\usepackage[french]{babel}
\usepackage[latin1]{inputenc}
\usepackage{times}
\usepackage[T1]{fontenc}
\usepackage{color}
\usepackage{hyperref}

\usepackage{listings}
\usepackage{tikz}
\usetikzlibrary{arrows}
%\tikzstyle{block}=[draw opacity=0.7,line width=1.4cm]


\pgfdeclareimage[height=1.2cm]{codexlogo}{codexlogo}
\pgfdeclareimage[height=5cm]{4thdim}{fourth-paradigm-cover}
\pgfdeclareimage[height=1.2cm]{leologo}{leo}

\newcommand {\mvd}{\mbox{$\; \rightarrow \! \! \! \! \rightarrow \; $}}

% The main document

\begin{document}

%%%%%%%%%%%%%%%%%%%%%%%%%%%%%%%%%%%%%%%%%%%%%%%%
%
%%%%%%%%%%%%%%%%%%%%%%%%%%%%%%%%%%%%%%%%%%%%%%%%%%%
\begin{frame}
  \titlepage
 \end{frame}

\begin{frame}{Big Data: goals and research interests}

\begin{block}{What are our goals?}
$\bullet$ Write a survey paper (even if there are several about this subject)\\
$\bullet$ An overview for finding interesting research points.\\
$\bullet$ Find collaborations: better understanding on the connexions of different domains involved in the theme\\ 
\end{block}

\begin{block}{Different research interests?}
$\bullet$ Parallelization: split a  particular task into  tasks executing concurrently on independent data sets and cooperating to compute final result.
Take advantage of  distributed resources; deal with system failures.\\
$\bullet$ Database: (declarative) language for querying, transforming and analysing big data.



\end{block}

\end{frame}
%%%%%%%%%%%%%%%%%%%%%
\begin{frame}

\small
\begin{block}{Compilation and Synthesis in Big Data Analytics - Christoph Koch}
Databases and compilers are two long-established and quite distinct areas of computer science. \textbf{With the advent of the big data revolution, these two areas move closer, to the point that they overlap and merge.} \\

Researchers in programming languages and compiler construction want to take part in this revolution, and also have to respond to the need of programmers for suitable tools to develop data-driven software for data-intensive tasks and analytics. \\

Database researchers cannot ignore the fact that most big-data analytics is performed in systems such as Hadoop that \textbf{run code written in general-purpose programming languages rather than query languages}. \\

\textbf{To remain relevant, each community has to move closer to the other.} 
\end{block}
\end{frame}


%%%%%
\begin{frame}{Trying to divide the theme into chapters (for discussion)}


\begin{enumerate}
\item MapReduce/Hadoop: notions and limits (Haloop: iterative data processing)
\item Query languages (PigLatin, NSQL)
\item  Big Graphs (Pregel: a system for large scale graph processing)
\item Languages for querying big social data (SocialLite, Datalog-like proposals...)
\item Parallel \& distributed databases: Scaling, partitioning, replication, massively parallel joins.
\item Concurrency control: transactions.
\item Eventual consistency.
\item ....
\end{enumerate}

\end{frame}
 %%%%%%%%%%%%%%%%%%
 
 
 
 \end{document}