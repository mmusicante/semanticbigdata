\documentclass{beamer}

\newcommand \com[1]{}


\usecolortheme{beaver}
\usetheme{CambridgeUS}
%\usetheme{Boadilla}

\title[SWANS-Dec2014]{A general Datalog-based framework for tractable query answering over ontologies}
%\subtitle{subtitle}
\author[Mirian Halfeld Ferrari]{Cali, Gottlob and Lukasiewicz}
\institute[LIFO]{Main Ideas of the paper}
\date{December 2014}

\setbeamerfont{block title}{size={}}
\setbeamercolor{titlelike}{parent=structure,bg=white}

\setbeamerfont*{frametitle}{size=\normalsize,series=\bfseries}
\setbeamertemplate{navigation symbols}{}
\setbeamertemplate{blocks}[shadow=true]

% Standard packages
\usepackage[french]{babel}
\usepackage[latin1]{inputenc}
\usepackage{times}
\usepackage[T1]{fontenc}
\usepackage{color}
\usepackage{hyperref}

\usepackage{listings}
\usepackage{tikz}
\usetikzlibrary{arrows}
%\tikzstyle{block}=[draw opacity=0.7,line width=1.4cm]

\pgfdeclareimage[height=1.2cm]{codexlogo}{codexlogo}
\pgfdeclareimage[height=5cm]{4thdim}{fourth-paradigm-cover}
\pgfdeclareimage[height=1.2cm]{leologo}{leo}

\def\A{\hbox{${\cal{A}}$}}
\def\K{\hbox{${\cal{K}}$}}
\def\R{\hbox{${\cal{R}}$}}
\def\T{\hbox{${\cal{T}}$}}


\def\dl{\hbox{\textsc{DL-lite}}\xspace}
\def\dla{\hbox{\textsc{DL-lite}$_{\cal A}$}\xspace}
\def\dlf{\hbox{\textsc{DL-lite}$_{\cal F}$}\xspace}
\def\dlr{\hbox{\textsc{DL-lite}$_{\cal R}$}\xspace}



\def\ie{\textit{i.e.}}
\def\eg{\textit{e.g. }}
\def\wrt{\textit{w.r.t. }}


\begin{document}


%%%%%%%%%%%%%%%%%%%%%%%%%%%%%%%%%%%%%%%%%%%%%%%%
%COMPLEXITE
%%%%%%%%%%%%%%%%%%%%%%%%%%%%%%%%%%%%%%%%%%%%%%%%%%%
\begin{frame}
  \titlepage
 \end{frame}
 
 \small
 \section{Introduction}
 
 \begin{frame}{Goal}
 \begin{block}{Since datalog is not as expressive as Description Logic (DL) ... :}
 \begin{itemize}
 \item  What are the main modifications of datalog, required for ontological query-answering ?
 \item Are there versions of datalog that encompass the DL-Lite family and that share the favourable data complexity bounds for query-answering with DL-Lite ?
 \end{itemize}

 \end{block} 

\end{frame}
 
 
 \section{Background}
 
 %%%%%%%%%%%%%%%%
 \begin{frame}{Guarded Datalog$^\pm$}
 \begin{block}{Extension}
 Existentially  qualified variables in rule heads.\\
 Expressive enough to model ontologies
  \end{block}
  
  
  \begin{block}{Tuple generating dependencies (TGD)}
  \begin{itemize}
  \item  Given a relational database schema \R, a TGD  $\sigma$ is a first-order formula of the form
  
  \centerline{$\forall \bf{X} \forall \bf{Y} ~\phi(\bf{X}, \bf{Y})  \rightarrow \exists \bf{Z} ~\psi (\bf{X}, \bf{Z})$}
  
  where $\phi(\bf{X}, \bf{Y})$  and  $\psi (\bf{X}, \bf{Z})$  are conjunctions of atoms over \R (without nulls).
  \item $\sigma$ is satisfied  in a database $D$ for \R\ iff whenever there exists a homomorphism $h$ that maps the atoms of $\phi(\bf{X}, \bf{Y})$
  to atoms of $D$, there exists an extension $h'$ of  $h$ that maps the atoms of  $\psi (\bf{X}, \bf{Z})$ to atoms of $D$.
  \item  We usually omit the universal quantifiers. All sets of TGD are finite here.
  
  \end{itemize}
  
  \end{block}
 
 \end{frame}
 
 %%%%%%%%%%%%%%%%
 \begin{frame}{Example}
 \begin{block}{Database $D$}
 \tiny
 employee(jo),
 manager(jo),
 directs(jo,finance),
 supervises(jo,ada),
 employee(ada),
 works\_in(ada,finance)
 \end{block}
 
 
 \begin{block}{Some constraints encoded as TGD}
 \tiny
\begin{enumerate}
\item  $manager(M) \rightarrow employee(M)$ ~~~ \textcolor{blue}{Every manager is an employee}
 \item $manager(M) \rightarrow \exists P~ directs(M, P)$ ~~ \textcolor{blue}{Every manager directs at least one department}
 \item $employee(E), directs(E,P) \rightarrow \exists E' ~ manager(E), supervises (E, E'), works\_in(E',P)$\\
 \textcolor{blue}{Every employee who directs a department is a manger, and supervises at least another employee who works in the same department}
 
 \item $employee(E), supervises(E, E'), manager(E') \rightarrow manager(E)$\\
 \textcolor{blue}{Every employee supervising a manager is a manager}
 
 \end{enumerate}
 \end{block}
 
 \begin{block}{Satisfaction}
 Above TGD satisfy $D$ but do not satisfy $D' = D \cup \{manager (ada)\}$ (2nd TGD)
 \end{block}
 \end{frame}
 
 
 \small
 %%%%%%%%%%%%%%%%
 \begin{frame}{Query answering under TGD}
 
 \tiny
 Let $D$ be a db for \R\, and let   $\Sigma$  be  a set of TGD  on \R.
 
 \begin{block}{Models}
% For a db $D$ for \R\, and a set of TGD $\Sigma$ on \R, 
$mods(D, \Sigma)$ (the set of models of $D$ and $\Sigma$) is the set of all (possibly infinite) databases $B$ s.t.:
 \begin{enumerate}
 \item $D \subseteq B$
 \item every $\sigma \in \Sigma$ is satisfied in $B$
 \end{enumerate}
 
 \end{block}
 
 \begin{block}{Evaluation of CQ (conjunctive queries)}
  $ans(Q,D, \Sigma)$ (the set of answers for a CQ $Q$ to $D$ and $\Sigma$) is the set of all tuples \textbf{a} s.t.
 $\textbf{a} \in Q(B)$ for all $B \in mods (D, \Sigma)$ 
 \end{block}

 
 \begin{block}{Evaluation of  BCQ (boolean conjunctive queries)}
 
 The answer of a BCQ $Q$ to $D$ and $\Sigma$ is YES (i.e, $D \cup \Sigma \models Q$) iff $ans(Q,D,\Sigma) \neq \emptyset$
 
 \end{block}
 
 \begin{block}{Complexity}
 Query answering under general TGD is \textcolor{red}{undecidable}
 \end{block}
 
 
 \end{frame}
 
 %%%%%%%%%%%%%%%%
 \begin{frame}{Guarded TGD}
 
 \begin{itemize}
 \item We should restrict rule syntax for achieving  decidability
 \item Rule bodies of TGD are guarded: in each rule body of a TGD there must exist an atom, called guard, in which all non-existentially quantified variables of the rule
 occur as argument.
 
 \end{itemize}
 
 \begin{center}
 $P(X), R(X,Y) , Q(Y) \rightarrow \exists Z~~ R(Y,Z)$
 \end{center}
 
 \begin{block}{Expression power}
 More expressive than DL-Lite, and so we are going to make more restrictions...
 \end{block}
 
 \end{frame}
 %%%%%%%%%%%%%%%%
 \begin{frame}{TGD Chase}
 \tiny
 \begin{block}{Definition}
 Procedure  for repairing a db relative to a set of dependencies, so that the result of the chase satisfies the dependencies.
 \end{block}
 
 \begin{block}{TGD Chase rule}
 We consider:
 \begin{itemize}
 \item A db $D$ over a relation schema \R
 \item A TGD $\sigma$ on \R\ of the form $ \forall \bf{Y} ~\phi(\bf{X}, \bf{Y})  \rightarrow \exists \bf{Z} ~\psi (\bf{X}, \bf{Z})$
 \end{itemize}
 Then $\sigma$ is applicable to $D$ if there exists a homomorphism $h$ that maps the atoms of $\phi(\bf{X}, \bf{Y})$
  to atoms of $D$.\\
  
  \vspace{0.3cm}
  Let $\sigma$ be applicable to D, and $h_1$ be a homomorphism that extends $h$ as follows\,: 
  
  for each $X_i \in \bf{X},$ $ h_1(X_i) = h(X_i)$; \\
  for each $Z_j \in  \bf{Z}, $  $h_1(Z_j) = z_j$,    where $z_j$ is a \textit{fresh} null, i.e., $z_j \in \Delta_ N , z_j$ does not occur in $D$, and $z_j $
  lexicographically follows all other nulls already introduced. 
  
  \vspace{0.3cm}
  The application of $\sigma$ on $D$ adds to $D$ the atom $h_1( (X,Z))$ if not already in $D$ (which is possible when $Z$ is empty).
 \end{block}
 
 \begin{block}{Chase algorithm}
  An exhaustive application of the TGD chase rule in a breadth-first (level-saturating) fashion, which leads as result to a (possibly infinite) chase for $D$ and $\Sigma$.
 \end{block}
 
 \end{frame}
 
 %%%%%%%%%%%%%%%%
 \begin{frame}{Equality-generating dependencies (EGD)}
 
 \begin{block}{EGD $\sigma$}
 \begin{itemize}
\item  A first-order formula of the form\\
 $$\forall \bf{X} ~\phi(\bf{X}) \rightarrow X_i = X_j$$\\
where  the body $\phi(\bf{X})$ is a (not necessarily guarded) conjunction of atoms (without nulls), and $X_i$ and $X_j$ are variables from $\bf{X}$. 

\item The head $X_i = X_j$  is satisfied in a database $D$ for \R\ iff, whenever there exists a homomorphism h such that 
$\phi(\textbf{X}) \subseteq D$, it holds that $h(X_i) = h(X_j)$. 

\item We usually omit the universal quantifiers in EGDs, and all sets of EGDs are finite here.
\end{itemize}
 \end{block}
 \end{frame}
 
 
 
 %%%%%%%%%%%%%%%%
 \begin{frame}{Example}
 
 \end{frame}
 
 
 \section{Datalog$^\pm_0$}
 %%%%%%%%%%%%%%%%
 \begin{frame}{Datalog$^\pm_0$}
 
 \begin{block}{Datalog$^\pm_0$ can be called a DL}
 \begin{itemize}
 \item Strictly more expressive than any of the DL-Lite family
 \item Linear TGD alone can express relationships such as  $ manager (X) \rightarrow manages (X,X)$ that are not expressible in any DL of DL-Lite family.
 \end{itemize}
 \end{block}
 
 \begin{block}{Characteristics}
 \begin{enumerate}
 \item Linear TGD
 \item Negative constraints
 \item Non-conflicting keys
 \end{enumerate}
 
 \end{block}
 
 \end{frame}
 
 
 %%%%%%%%%%%%%%%%
 \begin{frame}{Characteristics (1)}
 
 \begin{block}{Linear TGD}
 Only singleton body atom
 
 \begin{center}
 $\forall \bf{X} \forall \bf{Y}  \phi (\bf{X} \bf{Y}) \rightarrow  \exists \bf{Z} ~\psi (\bf{X}, \bf{Z})$
 \end{center}

 \end{block}
 
\begin{block}{Negative constraints}
Body is a conjunction of atoms (without nulls) not necessarily guarded 

\begin{center}
 $\forall \bf{X} \  \phi (\bf{X}) \rightarrow \bot$
 \end{center}
 
 Also written as 
 \begin{center}
 $\forall \bf{X}  \phi' (\bf{X} ) \rightarrow  \neg p (\bf{X})$
 \end{center}
 
 where $\phi' $ is obtained from $\phi$ by removing the atom $p (\bf{X})$.
 
\end{block} 
 
 \end{frame}
 
 
 %%%%%%%%%%%%%%%%
 \begin{frame} {Characteristics (2)}
 
 \begin{block}{Non conflicting keys}
 Let $K$ be a key.\\
 Let $\sigma$ be a TGD $\forall \bf{X} \forall \bf{Y}  \phi (\bf{X} \bf{Y}) \rightarrow  \exists \bf{Z} ~r(\bf{X}, \bf{Z})$\\
 $K$ is non conflicting with $\sigma$ iff either:
 \begin{itemize}
 \item the relational  predicate on which $K$ is defined is different from $r$, or
 \item the position of $K$ in $r$ are not a proper subset of the $\bf{X}$-position in $r$ in the head of $\sigma$ and
 \item  every variable in $\bf{Z }$ appears only once in the head of $\sigma$
 \end{itemize}
 \end{block}
 
 \begin{block}{Example}
 \tiny
 Four keys $k_1, k_2, k_3, k_4$  defined by the key attribute set 
 $K_1 = \{r[1], r[2]\}$, 
 $K_2 = \{r[1], r[3]\}$,
 $K_3 = \{r[3]\}$ and
 $K_4 = \{r[1]]\}$.
 
 TGD $\sigma = p(X,Y) \rightarrow \exists Z  ~r(X, Y, Z)$
 
 Then, the head predicate of $\sigma$ is $r$ and the set of positions in $r$  with universally quantified variables is $\bf{H}$ $= \{r[1], r[2]\}$.
 
 All keys but $K_4$ are NC with $\sigma$, since only $K_4 \subseteq $ $\bf{H}$
 
 \textcolor{red}{Every atom added in a chase by applying $\sigma$ would have a fresh null in some position in $K_1, K_2,$ and $ K_3$ and thus never firing the keys $k_1, k_2, k_3$}
 
 
 \end{block}
 
 \end{frame}
 
  %%%%%%%%%%%%%%%% %%%%%%%%%%%%%%%% %%%%%%%%%%%%%%%% %%%%%%%%%%%%%%%%
 \section{Datalog$^\pm_0$ and \dl}
 %%%%%%%%%%%%%%%% %%%%%%%%%%%%%%%% %%%%%%%%%%%%%%%% %%%%%%%%%%%%%%%%
 
 
  %%%%%%%%%%%%%%%%
 \begin{frame}{\dl}
 
 \begin{block}{Elementary ingredients}
 Let $A \in$   \textbf{C} be an atomic concept name, 
$P \in $ \textbf{C} be an atomic role and $P^-$ the inverse of  $P$.
In  the abstract syntax:

\begin{center}
\begin{tabular}{lcl}
$ B :: =  A \mid \exists R$ 	&\hspace{1cm}		&$ R ::= P \mid P^-$\\
$C ::= B \mid \neg B$ 		& 							&$ E ::= R \mid \neg R$ \\
\end{tabular}
\end{center}

\noindent
 $B$ denotes a \textit{basic concept}, \ie,  an atomic concept or a concept of 
 the form $\exists R$ and $R$ denotes a \textit{basic role}, \ie, a role that is either an 
 atomic role or the inverse of an atomic role. 
  Finally, $C$ denotes a \textit{}(general) concept, which can be a basic concept or its negation, 
  whereas $E$ denotes a \textit{(general) role}, which can be a basic role or its negation.
  
  \vspace{1cm}
  A  knowledge base ($\K = (\T,\A)$)  has two components: a TBox $\T$, used to represent intentional knowledge, and a ABox $\A$, used
to represent  extensional knowledge.
  \end{block}
 \end{frame}
 
  %%%%%%%%%%%%%%%%
 \begin{frame}{DL-Lite Family}
 
 \begin{block}{DL-Lite Family}
 \begin{itemize}
 \item ~\dlf: no role inclusion
 \item ~\dlr: no functionality constraints
 \item ~\dla: no functionality constraints on roles involved in role inclusions
 \end{itemize}
 
 \end{block}
 
 \url{http://webdam.inria.fr/Jorge/files/slquery-onto.pdf}
 \end{frame}
 
 
 
 
  %%%%%%%%%%%%%%%%
 \begin{frame}{Reduction to datalog$^\pm_0$}
 
 \begin{block}{Elementary Ingredients}
 \begin{itemize}
 \item Data value
 \item Data type
 \item Atomic concept
 \item Abstract role
 \item Attribute
 \item Individual
 \item Role attribute
 \end{itemize}
 
 \end{block}
 
 \end{frame}
 
 
  %%%%%%%%%%%%%%%%
 \begin{frame}{Translation of elementary ingredients}
 
 \small
 \begin{itemize}
 \item Every data value $v$ has a constant $\tau (v) = c \in \Delta$ s.t.  the $\tau (\bf{V}_d)$'s  for all datatypes $d \in \bf{D}$ are pairwise disjoint.
 \item  Every data type $d \in \bf{D}$ has under $\tau$  a predicate $\tau (d) = p_d$ along with the constraint $p_d(X) \wedge p'_d(X) \rightarrow \bot$
 for all pairwise  distinct $d, d' \in \bf{D}$.
 \item Every atomic concept $A \in \bf{A}$ has a unary predicate $\tau (A) = p_A \in \R$.
 \item Every abstract role $P \in \bf{R}_A$ has a binary predicate $\tau (P) = p_p \in \R$.
 \item Every attribute   $U \in \bf{R}_D$ has a binary predicate $\tau (U) = p_U \in \R$.
 \item Every individual $i \in \bf{I}$ has a constant $\tau (i) = c_i \in \Delta \setminus \bigcup_{d \in \bf{D}}$ $\tau (\bf{V}_d)$.  
 \textcolor{red}{$\Longleftarrow$ distinction  between data values and individuals !!! necessary ?}
 \end{itemize}
 \end{frame}
 
  %%%%%%%%%%%%%%%%
 \begin{frame}{Translation of axioms (concept inclusion)}
 

 

 \tiny
 Every concept inclusion axiom $B \sqsubseteq C$ is translated to the TGD or constraint $\tau (B \sqsubseteq C) = \tau'(B) \rightarrow \tau''(C)$ where
 \begin{itemize}
  \tiny
 \item $ \tau'(B) $ is defined as
 \begin{enumerate}
  \tiny
 \item $ p_A (X)$ if $B$ is of the form $A$,
  \item $ p_P (X,Y)$ if $B$ is of the form $\exists P$,
    \item $ p_P (Y,X)$ if $B$ is of the form $\exists P^-$,
    \item $ p_U (X,Y)$ if $B$ is of the form $\delta (U)$,  \textcolor{red}{$\Leftarrow$ concept attribute}
    \item $p_{U_R}(X,Y,Y')$ if $B$ is of the form $\exists \delta (U_R)$,  \textcolor{red}{$\Leftarrow$ role attribute}
    \item $p_{U_R}(Y,X,Y')$ if $B$ is of the form $\exists \delta (U_R)^-$,       \textcolor{red}{$\Leftarrow$ role attribute}
 
 \end{enumerate}

 \item $ \tau''(C) $ is defined as
 \begin{enumerate}
 \tiny
 \item $ p_A (X)$ if $C$ is of the form $A$,
  \item $ \exists Z ~ p_P (X,Z)$ if $C$ is of the form $\exists P$,
  \item  $ \exists Z ~ p_P (Z,X)$  if $C$ is of the form $\exists P^-$,
  \item $\exists Z ~   p_U (X,Z)$ if $C$ is of the form $\delta (U)$,
 \item $ \neg p_A (X)$ if $C$ is of the form $\neg A$, 
 \item $ \neg p (X,Y')$ if $C$ is of the form $\neg \exists P$,
\item $ \neg p (Y',X)$ if $C$ is of the form $\neg \exists P$,
 \item $\neg  p_U (X,Y')$ if $C$ is of the form $\neg \delta (U)$,
 \item $\exists Z ~   p_P (X,Z) \wedge p_A(Z)$ if $C$ is of the form $\exists P. A$,  \textcolor{red}{$\Leftarrow$ See!}
   \item $\exists Z ~   p_P (Z,X) \wedge p_A(Z)$ if $C$ is of the form $\exists P^{-}. A$. \textcolor{red}{$\Leftarrow$ See!}
  \item $\exists Z, Z' ~p_{U_R}(X,Z,Z')$ if $C$ is of the form $\exists \delta (U_R)$,  \textcolor{red}{$\Leftarrow$ role attribute}
   \item $\exists Z, Z' ~p_{U_R}(Z,X,Z')$ if $C$ is of the form $\exists \delta (U_R)^-$,  \textcolor{red}{$\Leftarrow$ role attribute}
    \item $\neg p_{U_R}(X,Z,Z')$ if $C$ is of the form $\neg \exists \delta (U_R)$,  \textcolor{red}{$\Leftarrow$ role attribute}
      \item $\neg p_{U_R}(Z,X,Z')$ if $C$ is of the form $\neg \exists \delta (U_R)^-$,  \textcolor{red}{$\Leftarrow$ role attribute}
 \end{enumerate}
 
  \end{itemize}

 \end{frame}
 
 
 %%%%%%%%%%%%%%%%
 \begin{frame}{Translation of axioms (functionality)}
 
 \begin{block}{Functionality axioms}
 \small
 \begin{itemize}
 \item Axiom (\textsf{func P}) is translated to the EGD $p_P(X,Y) \wedge p_P(X,Y') \rightarrow Y = Y'$
 \item Axiom (\textsf{func P$^-$}) is translated to the EGD $p_P(X,Y) \wedge p_P(X',Y) \rightarrow X = X'$
 \item  Axiom (\textsf{func U}) is translated to the EGD $p_U(X,Y) \wedge p_U(X,Y') \rightarrow Y = Y'$
 \item Axiom (\textsf{func $U_R$}) is translated to the EGD $p_{U_R}(X,Y,Z) \wedge p_{U_R}(X,Y, Z') \rightarrow Z = Z'$ \textcolor{red}{$\Leftarrow$ role attribute}
 \end{itemize}
 \end{block}
 
 \end{frame}
 
  %%%%%%%%%%%%%%%%
 \begin{frame}{Translation of axioms (membership)}
 
  \begin{block}{Membership axioms}
 \small
 \begin{itemize}
 \item Concept membership $A(a)$ is translated to $p_A(c_a)$,
 \item Role membership $P(a,b)$  is translated to $p_P(c_a, c_b)$,
 \item Attribute membership $U(a,v)$  is translated to $p_U(c_a, c_v)$,
 \item Role attribute membership $U_R(a,b,c)$ is translated to $p_{U_R}(c_a, c_b, c_c)$.
 \end{itemize}
 \end{block}
 
 \end{frame}
 
 
 
  %%%%%%%%%%%%%%%%
 \begin{frame}{Translation of axioms (role inclusion)}
  
  
  \begin{block}{Role inclusion}
  \small
 Every role inclusion axiom $Q \sqsubseteq R$ is translated to the TGD or constraint $\tau (Q \sqsubseteq R) = \tau'(Q) \rightarrow \tau''(R)$ where
 
 \begin{itemize}
 
  \small
 \item $ \tau'(Q) $ is defined as:

 \begin{itemize}
  \tiny
 \item  $p_P(X,Y)$ if $Q$ is of the form $P$,
 \item  $p_P(Y,X)$ if $Q$ is of the form $P^-$,
 \item $p_{U_R}(X, Y, Y')$ if $Q$ is of the form $\delta(U_R)$, \textcolor{red}{$\Leftarrow$ role attribute}
  \item $p_{U_R}(Y, X, Y')$ if $Q$ is of the form $\delta(U_R)^-$\textcolor{red}{$\Leftarrow$ role attribute}
 \end{itemize}
 
 \item $ \tau''(R) $ is defined as:
 
 \begin{itemize}
  \tiny
 \item  $p_P(X,Y)$ if $R$ is of the form $P$,
 \item  $p_P(Y,X)$ if $R$ is of the form $P^-$,
  \item  $\neg p_P(X,Y)$ if $R$ is of the form $\neg P$,
 \item  $\neg p_P(Y,X)$ if $R$ is of the form $\neg P^-$,
  \item $\exists Z~ p_{U_R}(X, Y, Y')$ if $R$ is of the form $\delta(U_R)$, \textcolor{red}{$\Leftarrow$ role attribute}
 \item $\exists Z~ p_{U_R}(Y, X, Y')$ if $R$ is of the form $\delta(U_R)^-$ \textcolor{red}{$\Leftarrow$ role attribute}
   \item $\neg \exists Z~ p_{U_R}(X, Y, Y')$ if $R$ is of the form $\neg \delta(U_R)$, \textcolor{red}{$\Leftarrow$ role attribute}
    \item $\neg \exists Z~ p_{U_R}(Y, X, Y')$ if $R$ is of the form $\neg \delta(U_R)^-$
 \textcolor{red}{$\Leftarrow$ role attribute}
 \end{itemize}
 
 \end{itemize}
 
 \end{block}
 
 \end{frame}
 
 
 
  %%%%%%%%%%%%%%%%
 \begin{frame}{Translation of axioms (attribute  inclusion)}
  
  
  \begin{block}{Attribute inclusion axiom}
  \begin{itemize}
  \tiny
 \item  $U \sqsubseteq U'$  is translated to the TGD  $p_U(X,Y) \rightarrow p_{U'} (X,Y)$,
 \item   $U \sqsubseteq \neg U'$ is  translated to the TGD $p_U(X,Y) \rightarrow \neg p_{U'} (X,Y)$
  \item  $U_R \sqsubseteq U'_R$  is translated to the TGD  $p_{U_R}(X,Y) \rightarrow p_{U'_R} (X,Y)$,
 \item   $U_R \sqsubseteq \neg U'_R$ is  translated to the TGD $p_{U_R}(X,Y) \rightarrow \neg p_{U'_R} (X,Y)$
 \end{itemize}
  \end{block}
  
   \end{frame}
 
 
 
  %%%%%%%%%%%%%%%%
 \begin{frame}{Translation of axioms (datatype  inclusion)}
  
  
  
  \begin{block}{Datatype inclusion axiom}
  \begin{itemize}
  \tiny
 \item Every datatype inclusion axiom  $\rho(U) \sqsubseteq d$  is  translated to the TGD  $p_U(X,Y) \rightarrow p_{d} (X,Y)$,
 \item datatype $p_U(X,Y) \sqsubseteq \top_D$ can be safely ignored.
 \item  Every datatype  inclusion axiom $E \sqsubseteq F$ is translated to  $\tau' (E)  \rightarrow \tau''(F)$ where
 \begin{itemize}
  \tiny
 \item $ \tau'(E) $ is defined as
  \begin{enumerate}
 \tiny
 \item $p_d(X)$ if $E$ is of the form $d$,
  \item $p_{U_R}(Y,Y', X)$ if $E$ is of the form $\rho(U_R)$
 \end{enumerate}
 
  \item and $  \tau'(F) $ is defined as
  
   \begin{enumerate}
 \tiny
 \item $p_d(X)$ if $E$ is of the form $d$,
  \item $p_{U_R}(Z,Z', X)$ if $E$ is of the form $\rho(U_R)$,
  \item $\neg p_d(X)$ if $E$ is of the form $\neg d$,
   \item $\neg p_{U_R}(Z,Z', X)$ if $E$ is of the form $\neg \rho(U_R)$,
 \end{enumerate}
 
 \end{itemize}
 \end{itemize}
  \end{block}
 
 \end{frame}

 
 
 
  %%%%%%%%%%%%%%%%
 \begin{frame}
 
 \end{frame}
 
 
 

 \end{document}